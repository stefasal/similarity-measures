\documentclass[10pt,english]{article}
\usepackage[T1]{fontenc}
\usepackage[utf8]{inputenc}
\usepackage{amsmath}
\usepackage[margin=1in]{geometry}
\usepackage{listings}
\usepackage{graphicx}
\usepackage{float}
\usepackage{color}
\definecolor{dkgreen}{rgb}{0,0.6,0}
\definecolor{gray}{rgb}{0.5,0.5,0.5}
\definecolor{mauve}{rgb}{0.58,0,0.82}

\lstset{frame=tb,
language=R,
aboveskip=3mm,
belowskip=3mm,
showstringspaces=false,
columns=flexible,
numbers=none,
keywordstyle=\color{blue},
alsoletter={.},
numberstyle=\tiny\color{gray},
commentstyle=\color{dkgreen},
stringstyle=\color{mauve},
breaklines=true,
breakatwhitespace=true,
tabsize=3
}

\renewcommand{\vec}[1]{\mathbf{#1}}
\newcommand{\problem}[1]{\subsection*{#1}}
\newcommand{\subproblem}[1]{\subsubsection*{#1)}}
\newcommand{\subsubproblem}[1]{\paragraph*{#1}}
\newcommand{\rscript}[1]{\lstinputlisting[language=R]{#1.r}}
\newcommand{\xni}[1]{\bar{x}_{#1\cdot}}
\newcommand{\Xni}[1]{\bar{X}_{#1\cdot}}
\newcommand{\exfig}[1]{
\begin{figure}[H]
\includegraphics{#1}
\end{figure}
}
\newcommand{\avg}[1]{\bar{#1}}
\newcommand{\med}[1]{med{x}}
\newcommand{\wave}[1]{\tilde{#1}}
\newcommand{\dx}[0]{dx}

\newcommand{\pardiff}[2]{\frac{\partial #1}{\partial #2}}
\newcommand{\normd}[3]{\frac{1}{\sqrt{2\pi #3 }}e^{-(#1 - #2 )^2/2 #3 }}
\newcommand{\cnormd}[2]{\frac{1}{\sqrt{2\pi #2 }}e^{- #1^2/2 #2 }}
\newcommand{\snormd}[1]{\frac{1}{\sqrt{2\pi}}e^{- #1^2/2}}
\newcommand{\invgamma}[3]{\frac{#3^#2}{\Gamma(#2)} #1^{-#2-1}e^{-#3/#1}}
\newcommand{\tinvgamma}[0]{\text{Inv-Gamma}}
\newcommand{\tlogit}[0]{\text{logit}}
\newcommand{\prop}{=}
\newcommand{\abs}[1]{\left|#1\right|}

%%% Local Variables:
%%% mode: latex
%%% TeX-master: t
%%% End:


\newcommand{\forbes}[2]{\frac{\abs{#1 \cap #2}}{\abs{#1}\abs{#2}}}
\newcommand{\jaccard}[2]{\frac{\abs{#1 \cap #2}}{\abs{#1 \cup #2}}}
\begin{document}
\problem{Hypergeometric}
Let $R$ be created by sampling $n_1$ points from $X$ and $n_2$ points from $X$. For the Forbes measure we then get:
\begin{align*}
  \abs{R} &= n_1+n_2\\
  FDR &= n_2/(n_1+n_2)\\
  \abs{R \cap Q} &\sim HG(\abs{X}, \abs{X \cap Q}, n_1) + HG(\abs{\bar{X}}, \abs{\bar X \cap Q}, n_2)\\
  E(\forbes{R}{Q}) &= \frac{n_1\abs{X \cap Q}/\abs{X}+n_2\abs{\bar X \cap Q}/\abs{\bar{X}}}{\abs{Q}(n_1+n_2)}\\
          &= (1-FDR)\forbes{X}{Q} + FDR\forbes{\bar{X}}{Q}\\
\end{align*}
I.e:  the expected Forbes measure for $R$ and $Q$ is a weighted linear combination of the forbes similarity between $X$ and $Q$ and $\bar X$ and $Q$. But independent of the size of $R$.

For Jaccard we get:
\begin{align*}
  \abs{R \cap Q} &\sim HG(\abs{X}, \abs{X \cap Q}, n_1) + HG(\abs{\bar{X}}, \abs{\bar X \cap Q}, n_2)\\
  E(\abs{R \cap Q}) &=   HG(\abs{X}, \abs{X \cap Q}, n_1) + HG(\abs{\bar{X}}, \abs{\bar X \cap Q}, n_2)\\
  \jaccard{R}{Q} &= \frac{\abs Q + \abs R - \abs{Q \cup R}}{\abs{Q \cup R}}\\
  \jaccard{R}{Q} &\sim \frac{\abs{R \cap Q}}{\abs{Q}+\abs{R}-\abs{R \cap Q}}\\
\end{align*}

\problem{Notes}
\begin{align*}
  f(x) &= (x)/(N-x)\\
  f'(x) &= ((N-x)+x)/(N-x)^2\\
  f'(x) &= ((N)/(N-x)^2\\
\end{align*}
\end{document}

%%% Local Variables:
%%% mode: latex
%%% TeX-master: t
%%% End:
